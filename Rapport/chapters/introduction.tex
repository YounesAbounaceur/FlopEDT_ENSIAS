\pagenumbering{arabic}
\chapter*{Introduction}
\addcontentsline{toc}{chapter}{Introduction Générale}
Afin d’appliquer les méthodologies et les notions enseignées durant le cours d'Ingénierie du Web, nous sommes invités à réaliser un projet qui nous permet d’appliquer nos connaissances théoriques sur le champ pratique.\\

Dans toute établissement universitaire, bien organiser l'emplois du temps est l'une des tâches les plus prioritaires du directeur ou du responsable de fillière.

Notre projet se situe dans ce contexte, il s’agit d’une application web qui automatise le processus de génération de l'emplois du temps, ce qui va faciliter la vie pour le responsable de fillière.\\

Le présent rapport est dédié à la présentation de l'ensemble des travaux menés dans le cadre de notre projet. Le premier chapitre est destiné à la description du contexte général de ce projet, allant de l'amenant, la problématique, aux objectifs du projet.\\

Le deuxième chapitre aborde dans un premier temps une étude et une analyse des besoins, en commençant par l'étude de l'existant, ensuite l'analyse des fonctionnalités à implémenter. Puis, en proposant en deuxième lieu une conception technique de l'application.\\

Le dernier chapitre est destiné à décrire la mise en oeuvre du projet, il contient les outils choisis pour le développement ainsi que les détails des différentes phases de réalisation de l'application.\\

Enfin, la conclusion générale résume le bilan du travail effectué et les principales perspectives.\\
\newpage














%Ce rapport peut ainsi être subdivisé en cinq parties :
%\begin{itemize}
%\item kmkkkm
% \end{itemize}
% 