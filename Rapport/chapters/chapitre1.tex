\chapter{Contexte général du projet}
Dans ce chapitre nous allons définir le contexte général du projet et ses objectifs
\newpage

\section{Amenant}
L'ingénierie du web, est l'une des spécialités les plus demandées dans le monde du travail, et autant que étudiants 2éme année Géniel logiciel à l'ENSIAS, c'est une capacité qu'on doit maitriser, du coup, nous sommes menés à réaliser une application web en se  basant sur ce qu'on a vu dans le cours d'ingénierie web avec Mr EL Hamlaoui.\\


\section{Problématique}
Au sein de toute université dans le monde, les étudiants et les professeurs doivent respecter un emplois du temps. Or, la création de cet emplois du temps prend beacoup du temps et d'effort par la personne en charge, généralement c'est le chef de fillière. Ce dernier doit contacter chaque professeur et voir quand est ce qu'il est disponible, par la suite il doit créer un emplois qui respecte au maximum possible les contraintes du disponibilité des profs, et finalement il contacte les profs pour vérifier si l'emplois proposé est convenable pour eux.\\


\section{Analyse de besoin}
Afin de faciliter cette tâche pour le chef de fillière, et automatiser ce long processus qui inclut plusieurs intervenants, nous avons adapté et amélioré une application web \textbf{FlOp EDT} qui génére un emplois du temps qui satisfait les professeurs et qui est validé par le chef de fillière.\\

\section{Objectifs du projet}
Parmis les tâches pour réaliser un bon déroulement du projet:\\
\begin{itemize}
\item Adapter la base de données, pour qu'elle contienne les noms des profs, les salles et les modules de l'ENSIAS.
\item Améliorer le design et l'interface de l'application.
\item Améliorer la performance de l'application et fixer quelques problèmes fonctionnelles.
\item Visualisation des résultats.
\end{itemize}

